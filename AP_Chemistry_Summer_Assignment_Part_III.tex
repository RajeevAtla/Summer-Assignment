\documentclass[11pt]{scrartcl}
\usepackage[utf8]{inputenc}
\usepackage{mathtools}
\usepackage{amssymb}
\usepackage{fancyhdr}
\usepackage{answers}
\usepackage{lastpage}
\usepackage{datetime}
\usepackage{titlesec}
\usepackage{makeidx}
\usepackage{graphicx}
\usepackage{bm}
\usepackage{evan}[fancy, sexy, hdr, colorsec]
\usepackage[dvipsnames]{xcolor}
\usepackage[margin = 0.75in, letterpaper, portrait]{geometry}
\usepackage[version = 3.9]{mhchem}

\everymath{\displaystyle}

\pagestyle{fancy}
\rhead{Last Updated: \today\ \currenttime\ UTC}
\lhead{AP Chemistry Summer Assignment Part III}
\cfoot{Page \thepage\ of \pageref*{LastPage}}
\rfoot{\copyright\ $2020$ Rajeev Atla}

\renewcommand*\contentsname{\S Table of Contents}

\makeindex

\Newassociation{hint}{hintitem}{all-hints}
\renewcommand{\solutionextension}{out}
\renewenvironment{hintitem}[1]{\item[\bf #1.]}{}

\begin{document}

\titleformat{\section}{\normalfont\Large\bfseries}{\color{red}\S \thesection}{0.5em}{}
\titleformat{\subsection}{\normalfont\Large\bfseries}{\color{olive}\S \thesubsection}{0.5em}{}
\titleformat{\subsubsection}{\normalfont\Large\bfseries}{\color{blue}\S \thesubsubsection}{0.5em}{}

\begin{center}
    \Large \textbf{AP Chemistry Summer Assignment Part III}
\end{center}
\begin{center}
    \Large Rajeev Atla
\end{center}

Some beginning notes:
\begin{itemize}
    \item For personal use only
    \item Feel free to contact me ASAP at \href{mailto:rajeev@rajeevatla.com}{rajeev@rajeevatla.com} if there's anything wrong/confusing
    \item Due date: \textbf{First Day of School - 9/3}
    \item Some notation:
    \begin{itemize}
        \item $V$ is volume
        \item $r$ is radius
        \item $m$ is mass
        \item $\rho$ is density
        \item $\delta$ (delta) is error
        \item $n$ is moles
        \item $M$ is molar mass
    \end{itemize}
    \item \href{https://www.jpsaos.com/pittenger/APSummerReviewproblems.pdf}{See the original assignment on Mrs. Pittenger's website}
    \item For some help in chemistry, see your teacher, your textbook, or your local Science National Honor Society (SNHS) chapter
    \item Speaking of textbooks, the one used in this class was Zumdahl's \textit{Chemistry}
    \item \href{https://ptable.com}{Ptable} was used as the periodic table throughout this assignment
\end{itemize}

\newpage
\tableofcontents

\newpage


\section{Problem Set 1 - Math Basics}

\subsection{Problem 1}
How many significant figures are in each of the following?

\subsubsection{a}
\boxed{\text{$12$ has $2$ significant figures.}}

\subsubsection{b}
\boxed{\text{$10980$ has $4$ significant figures.}}

\subsubsection{c}
\boxed{\text{$2001$ has $4$ significant figures.}}

\subsubsection{d}
\boxed{\text{\(2.001 \times 10^3\) has $4$ significant figures.}}

\subsubsection{e}
\boxed{\text{$0.0000101$ has $3$ significant figures.}}

\subsubsection{f}
\boxed{\text{$1.01 \times 10^{-5}$ has $3$ significant figures.}}

\subsubsection{g}
\boxed{\text{$1000.$ has $4$ significant figures.}}

\subsubsection{h}
\boxed{\text{pH $2.1$ has $1$ significant figure.}}

\subsection{Problem 2}
Use scientific notation to express the number 480 to $\dots$

\subsubsection{(a) 1 Significant Figure}

\boxed{5 \times 10^2}

\subsubsection{(b) 2 Significant Figures}

\boxed{4.8 \times 10^2}

\subsubsection{(c) 3 Significant Figures}

\boxed{4.80 \times 10^2}

\subsubsection{(d) 4 Significant Figures}

\boxed{4.800 \times 10^2}

\newpage
\subsection{Problem 3}

 Perform the following mathematical operations and express each result to the correct number of significant
figures.

\subsubsection{(a)}

\begin{align*}
    97.381 \div 4.2502 + 0.99195 &= 22.912 + 0.99195 \\
    &= \boxed{23.904} \\
\end{align*}

\subsubsection{(b)}

\begin{align*}
    171.5 + 72.915 - 8.23 &= \boxed{236.2} \\
\end{align*}

\subsubsection{(c)}

\begin{align*}
    1.00914 \div 0.87104 + 1.2012 &= 1.1585 + 1.2012\\
    &= \boxed{2.3597}
\end{align*}

\subsubsection{(d)}

\begin{align*}
    21.901 - 13.21 - 4.0215 &= \boxed{4.67}
\end{align*}

\newpage
\subsection{Problem 4}
 Perform the following mathematical operations and express each result to the correct number of significant
figures.

\subsubsection{(a)}

\begin{align*}
    \left (0.102 \times 0.0821 \times 273 \right) \div 1.01 &= 2.29 \div 1.01 \\
    &= \boxed{2.27} \\
\end{align*}

\subsubsection{(b)}

\begin{align*}
    0.14 \times \left (6.022 \times 10^{23} \right) &= 0.84 \times 10^{23} \\
    &= \boxed{8.4 \times 10^{22}} \\
\end{align*}

\subsubsection{(c)}

\begin{align*}
    \left (4.0 \times 10^4 \right) \times \left (5.021 \times 10^{-3} \right) \times \left (7.34993 \times 10^{2} \right) &= 150 \times 10^{3} \\
    &= \boxed{1.5 \times 10^{5}}
\end{align*}

\subsubsection{(d)}

\begin{align*}
    \left (2.00 \times 10^{6} \right) \div \left (3.00 \times 10^{-7} \right) &= 0.667 \times 10^{13} \\
    &= \boxed{6.67 \times 10^{12}} \\
\end{align*}

\subsubsection{(e)}

\begin{align*}
    4.184 \times 100.62 \times \left(25.27 -24.16 \right) &= 421.0 \times 1.11 \\
    &= \boxed{467} \\
\end{align*}

\subsubsection{(f)}

\begin{align*}
    \left [ \left (8.925 - 8.904 \right) \div 8925 \right] \times 100 &= \left (0.021 \div 8925 \right) \times 100 \\
    &= \boxed{2.4 \times 10^{-4}}
\end{align*}

\subsubsection{(g)}

\begin{align*}
    \left (9.04 + 8.23 + 21.954 + 81.0 \right) \div 3.1416 &= 120.2 \div 3.1416 \\
    &= \boxed{38.26} \\
\end{align*}

\subsubsection{(h)}

\begin{align*}
    \left (9.2 \times 100.65 \right) \div \left (8.321 + 4.026 \right) &= 930 \div 12.347 \\
    &= \boxed{75} \\
\end{align*}

\subsubsection{(i)}

\begin{align*}
    0.6154 + 2.07 - 2.114 &= \boxed{0.57} \\
\end{align*}

\subsubsection{(j)}

\begin{align*}
    8.27 \left (4.987 - 4.962 \right) &= 8.27 \times 0.025 \\
    &= \boxed{0.21} \\
\end{align*}

\subsubsection{(k)}

Note that $4$ is exact because we are taking an average here.

\begin{align*}
    \left (9.5 + 4.1 + 2.8 + 3.175 \right) \div 4 &= 19.6 \div 4 \\
    &= \boxed{4.90} \\
\end{align*}

\subsubsection{(l)}
100 is exact here.

\begin{align*}
    \left [ \left (9.025 - 9.024 \right) \div 9.025 \right] \times 100 &= \left (0.001 \div 9.025 \right) \times 100 \\
    &= \boxed{0.01} \\
\end{align*}

\subsection{Problem 5}
The density of aluminum is $2.70\ \frac{\text{g}}{\text{cm}^3}$.
Express this value in units of kilograms per cubic meter and pounds per cubic foot.

\begin{align*}
    2.70\ \frac{\text{g}}{\text{cm}^3} \times \frac{1\ \text{kg}}{1000\ \text{g}} \times \left (\frac{100 \ \text{cm}}{1\ \text{m}} \right)^3 &= \boxed{2700\ \frac{\text{kg}}{\text{m}^3}} \\
\end{align*}

\begin{align*}
    2.70\ \frac{\text{g}}{\text{cm}^3} \times  \frac{1\ \text{lb}}{453.59237\ \text{g}} \times \left ( 30.48\ \frac{\text{cm}}{\text{ft}} \right)^3 &= \boxed{169\ \frac{\text{lbs}}{\text{ft}}}
\end{align*}

\newpage
\subsection{Problem 6}
 A material will float on the surface of a liquid if the material has a density less than that of the liquid.
 Given that the density of water is approximately $1.0 \frac{\text{g}}{\text{mL}}$, will a block of material having a volume of $1.2 \times 10^4 \ \text{in}^3$ and weighing $350$ lbs float or sink when placed in a reservoir of water?


\noindent First we find the density:
\begin{align*}
    \frac{350\ \text{lbs}}{1.2 \times 10^{4}\ \text{in}^3} &= 0.029\ \frac{\text{lbs}}{\text{in}^3} \\
    &= \left ( 0.029\ \frac{\text{lbs}}{\text{in}^3} \right) \times \left (453.59237\ \frac{\text{g}}{\text{lbs}} \right)  \times \left (0.0610237\ \frac{\text{in}^3}{\text{mL}} \right) \\
    &= \boxed{0.80\ \frac{\text{g}}{\text{mL}}} \\
\end{align*}

\newpage
\subsection{Problem 7}
A star is estimated to have a mass of $2 \times 10^{36}\ \text{kg}$.
Assuming it to be a sphere of average radius $7.0 \times 10^{5} \ \text{km}$, calculate the average density of the star in units of grams per cubic centimeter.


\noindent It's actually easier here to convert first, then do the math.
\begin{align*}
    m &= 2 \times 10^{36}\ \text{kg} = 2 \times 10^{39}\ \text{g} \\
    r &= 7.0 \times 10^{5}\ \text{km} = 7.0 \times 10^{10}\ \text{cm} \\
\end{align*}

\begin{align*}
    \rho &= \frac{m}{V} \\
    &= \frac{m}{\frac{4}{3} \pi r^3} \\
    &= \frac{3m}{4 \pi r^3} \\
    &= \frac{3 \left (2 \times 10^{39}\ \text{g} \right)}{4 \pi \left (7.0 \times 10^{10}\ \text{cm} \right)^3} \\
    &= \boxed{1 \times 10^{6}\ \frac{\text{g}}{\text{cm}^3}} \\
\end{align*}

\newpage
\subsection{Problem 8}

A rectangular block has dimensions $2.9\ \text{cm} \times 3.5\ \text{cm} \times 10.0\ \text{cm}$.
The mass of the block is $615.0\ \text{g}$.
What are the volume and density of the block?

\begin{align*}
    V &= 2.9\ \text{cm} \times 3.5\ \text{cm} \times 10.0\ \text{cm} = \boxed{1.0 \times 10^{2}\ \text{cm}^3} \\
    \rho &= \frac{ 615.0 \ \text{g} }{100\ \text{cm}^3} = \boxed{6.2\ \frac{\text{g}}{\text{cm}^3}} \\
\end{align*}

\newpage
\subsection{Problem 9}
Calculate the error percentage for each case:

\subsubsection{(a)}
The density of an aluminum block determined in experiment was $2.64\ \frac{\text{g}}{\text{cm}^3}$. The true value is $2.70\ \frac{\text{g}}{\text{cm}^3}$.

\begin{align*}
    \delta &= 100 \times \frac{\left |\rho_{\text{expected}} - \rho_{\text{actual}} \right| }{\rho_{\text{expected}}} \\
    &= 100 \times \frac{0.06}{2.70} \\
    &= \boxed{2.2\ \%} \\
\end{align*}

\subsubsection{(b)}
The experimental determination of iron in a sample of iron ore was $16.48\ \%$.
The true value was $16.12\ \%$.

\begin{align*}
    \delta &= 100 \times \frac{16.48-16.12}{16.12} \\
    &= 100 \times \frac{0.36}{16.12} \\
    &= \boxed{2.2\ \%} \\
\end{align*}

\newpage
\section{Problem Set 2 - Atoms, Ions, and Compounds}
\subsection{Problem 1}
You have a chemical in a sealed glass container filled with air. The system has a mass of $250.0\ \text{g}$. The
chemical is ignited by means of a magnifying glass focusing sunlight on the reactant. After the chemical
is completely burned, what is the mass of the setup? Explain your answer.

\noindent \boxed{250.0\ \text{g}}
Since the container is sealed, no matter can leave or enter the setup.
Hence, the final mass is the same as the initial mass.

\newpage
\subsection{Problem 2}
Find the empirical and molecular formulas of the following compounds.

\subsubsection{(a)}
$73.8 \%$ carbon, $8.7\ \%$ hydrogen, $17.5\ \%$ nitrogen, molar mass $166.0\ \frac{\text{g}}{\text{mol}}$

\noindent First, we assume that we are looking at a $100\ \text{g}$ sample, and so the percentages are equal to the mass in grams.
Next, we convert to moles.
\begin{align*}
    n_{\text{C}} &= \left ( 73.8\ \text{g} \right) \times \left ( \frac{1\ \text{mol}}{12.011\ \text{g}} \right) \\
    &= 6.14\ \text{mol} \\
    n_{\text{H}} &= \left ( 8.7\ \text{g} \right) \times \left ( \frac{1\ \text{mol}}{1.008\ \text{g}} \right) \\
    &= 8.6\ \text{mol} \\
    n_{\text{N}} &= \left ( 17.5\ \text{g} \right) \times \left ( \frac{1\ \text{mol}}{14.007\ \text{g}} \right) \\
    &= 1.25\ \text{mol} \\
\end{align*}
Next, we find the \textit{relative} molar values $N_i$ by dividing by the least number of moles, which in this case is $n_{\text{N}}$.

\begin{align*}
    N_{\text{C}} &\Rightarrow \frac{6.14\ \text{mol}}{1.25\ \text{mol}} = 4.91 \approx 5\\
    N_{\text{H}} &\Rightarrow \frac{8.6\ \text{mol}}{1.25\ \text{mol}} = 6.9 \approx 7 \\
    N_{\text{N}} &\Rightarrow \frac{1.25\ \text{mol}}{1.25\ \text{mol}} = 1 \\
\end{align*}

The empirical formula is then \boxed{\text{C}_5\, \text{H}_7\, \text{N}}.
The molar mass of the empirical molecule is

\begin{align*}
    M_{\text{e}} &= 5 \left ( \frac{12.011\ \text{g}}{\text{mol}} \right) + 7 \left ( \frac{1.008\ \text{g}}{\text{mol}} \right) + \left ( \frac{14.007\ \text{g}}{\text{mol}} \right) \\
    &= \left ( \frac{60.055\ \text{g}}{\text{mol}} \right) + \left ( \frac{7.056\ \text{g}}{\text{mol}} \right) + \left (\frac{14.007\ \text{g}}{\text{mol}} \right) \\
    &= 81.118\ \frac{\text{g}}{\text{mol}}\\
\end{align*}
The scaling factor to the molecular formula is then
$$\frac{81.118\ \frac{\text{g}}{\text{mol}}}{166.0\ \frac{\text{g}}{\text{mol}}} = 0.4887 \approx 0.5$$
This indicates that the empirical formula's coefficients need to be doubled to get the molecular formula.
The molecular formula is then \boxed{\text{C}_{10} \text{H}_{14} \text{N}_{2}}, meaning this compound is likely anabasine.

\subsubsection{(b)}
$80.0\ \%$ Carbon, $20.0\ \%$ Hydrogen, $30.0\ \frac{\text{g}}{\text{mol}}$ Molar mass

We start by assuming that we have a $100$ gram sample, so we just need to consider $80.0\ \text{g}$ Carbon and $20.0\ \text{g}$ Hydrogen.
Converting to moles,
\begin{align*}
    n_{\text{C}} = \left ( 80.0\ \text{g C} \right) \left (\frac{1\ \text{mol C}}{12.0107\ \text{g C}} \right) &= 6.66\ \text{mol C} \\
    n_{\text{H}} = \left (20.0\ \text{g H} \right ) \left ( \frac{1\ \text{mol H}}{1.00784\ \text{g H}} \right) &= 19.8\ \text{mol H}
\end{align*}
Next we find the relative molar ratios.
\begin{align*}
    N_\text{C} &= \frac{6.66}{6.66} = 1 \\
    N_{\text{H}} &= \frac{19.8}{6.66} = 2.97 \approx 3 \\
\end{align*}
This means that the empirical formula is \boxed{\text{CH}_3}.
However, the molar mass of this empirical compound is
\begin{align*}
    M &= \left ( \frac{12.011\ \text{g} }{\text{mol}} \right) + 3 \left ( \frac{1.008\ \text{g}}{\text{mol}} \right) \\
    &= \left ( \frac{12.011\ \text{g} }{\text{mol}} \right) + \left ( \frac{3.024\ \text{g}}{\text{mol}} \right) \\
    &= 15.035\ \frac{ \text{g}}{\text{mol}} \\
\end{align*}
Since this is approximately half the molecular molar mass, we must multiply the empirical coefficients by $\frac{1}{\frac{1}{2}} = 2$ to get the molecular formula, which is \boxed{\text{C}_2 \text{H}_6}, meaning that this molecule is ethane.

\newpage
\subsection{Problem 3}
In the periodic table, how many elements are found in \dots

\subsubsection{(a)}
the second period?

\boxed{8}

\subsubsection{(b)}
the third period?

\boxed{8}

\subsubsection{(c)}
group 2A?

\boxed{6}

\subsubsection{(d)}
the oxygen family?

\boxed{6}

\subsubsection{(e)}
the fourth period?

\boxed{18}

\subsubsection{(f)}
the nickel group?

\boxed{4}

\subsubsection{(g)}
group 8A?

\boxed{4}

\newpage
\subsection{Problem 4}
Give the number of protons and neutrons in each of the following atoms.

\subsubsection{(a)}
\ce{^{238}_{84} Pu}

\boxed{84} protons and \boxed{154} neutrons

\subsubsection{(b)}
\ce{^{65}_{29} Cu}

\boxed{29} protons and \boxed{36} neutrons

\subsubsection{(c)}
\ce{^{52}_{24} Cr}

\boxed{24} protons and \boxed{28} neutrons

\subsubsection{(d)}
\ce{^{60}_{27} Co}

\boxed{27} protons and \boxed{33} neutrons

\subsubsection{(e)}
\ce{^{52}_{24} Cr}

\boxed{24} protons and \boxed{28} neutrons

\subsubsection{(f)}
\ce{^{4}_{2} He}

\boxed{2} protons and \boxed{2} neutrons

\newpage
\subsection{Problem 5}
Complete the following table:

\begin{center}
    \begin{tabular}{||c|c|c|c|c||}
        \hline
        \hline
        \textbf{Symbol} & \textbf{Protons} & \textbf{Neutrons} & \textbf{Electrons} & \textbf{Net charge}\\
        \hline
        \hline
         & 33 & 42 & 30 & 3+ \\
        \hline
        \ce{^{128}_{52}Te^2-} &  &  & 54 &  \\
        \hline
        & 16 & 16 & 16 &  \\
        \hline
        & 81 & 123 &  & 1+ \\
        \hline
        \ce{^{195}_{78} Pt} &  &  &  &  \\
        \hline
        \hline
    \end{tabular}
\end{center}

\begin{center}
    \begin{tabular}{||c|c|c|c|c||}
        \hline
        \hline
        \textbf{Symbol} & \textbf{Protons} & \textbf{Neutrons} & \textbf{Electrons} & \textbf{Net charge}\\
        \hline
        \hline
        \ce{^{75}_{33}As^3+} & 33 & 42 & 30 & 3+ \\
        \hline
        \ce{^{128}_{52}Te^2-} & 52 & 76 & 54 & 2- \\
        \hline
        \ce{^{32}_{16}S}& 16 & 16 & 16 & 0 \\
        \hline
        \ce{^{123}_{81}Tl^1+}& 81 & 123 & 80 & 1+ \\
        \hline
        \ce{^{195}_{78}Pt} & 78 & 117 & 78 & 0 \\
        \hline
        \hline
    \end{tabular}
\end{center}

\newpage
\subsection{Problem 6}
Would you expect each of the following atoms to gain or lose electrons when forming ions?
Which ion is most likely to form in each case?

\subsubsection{(a) Ra}
Radium is most likely to form \boxed{\ce{Ra^2+}}, thereby loosing 2 electrons.

\subsubsection{(b) In}
Indium is most likely to form \boxed{\ce{In^3+}}, thereby loosing 3 electrons.

\subsubsection{(c) P}
Phosphorus is most likely to form \boxed{\ce{P^3-}}, thereby gaining 3 electrons.

\subsubsection{(d) Te}
Tellurium is most likely to form \boxed{\ce{Te^2-}}, thereby gaining 2 electrons.

\subsubsection{(e) Br}
Bromine is most likely to form \boxed{\ce{Br^1-}}, thereby gaining 1 electron.

\subsubsection{(f) Rb}
Rubidium is most likely to form \boxed{\ce{Rb}^{1+}}, thereby loosing 1 electron.

\newpage
\subsection{Problem 7}
Name each of the following compounds:

\begin{center}
        \begin{tabular}{||c|c|c||}
         \hline
         \hline
         \textbf{Index} & \textbf{Formula} & \textbf{Name}\\
         \hline
         \hline
         (a) & \ce{Rb_2O} & \\
         \hline
         (b) & \ce{AlI_3} & \\
         \hline
         (c) & \ce{FeBr_3} & \\
         \hline
         (d) & \ce{Hg_2O} & \\
         \hline
         (e) & \ce{CoS} & \\
         \hline
         (f) & \ce{TiCl_4} & \\
         \hline
         (g) & \ce{CrO_2} &  \\
         \hline
         (h) & \ce{Cr_2O_3} &  \\
         \hline
         (i) & \ce{NaH} &  \\
         \hline
         (j) & \ce{ZnCl_2} & \\
         \hline
         (k) & \ce{CsF} &  \\
         \hline
         (l) & \ce{Li_3N} &  \\
         \hline
         (m) & \ce{Ag_2S} &  \\
         \hline
         (n) & \ce{Sr_3P_2} &  \\
         \hline
         (o) & \ce{MnO_2} &  \\
         \hline
         \hline
    \end{tabular}
\end{center}

\begin{center}
        \begin{tabular}{||c|c|c||}
         \hline
         \hline
         \textbf{Index} & \textbf{Formula} & \textbf{Name}\\
         \hline
         \hline
         (a) & \ce{Rb_2O} & Rubidium (I) oxide \\
         \hline
         (b) & \ce{AlI_3} & Aluminum iodide \\
         \hline
         (c) & \ce{FeBr_3} & Iron (III) bromide \\
         \hline
         (d) & \ce{Hg_2O} & Mercury (I) oxide\\
         \hline
         (e) & \ce{CoS} & Cobalt (II) sulfide\\
         \hline
         (f) & \ce{TiCl_4} & Titanium (IV) chloride\\
         \hline
         (g) & \ce{CrO_2} & Chromium (IV) oxide \\
         \hline
         (h) & \ce{Cr_2O_3} & Chromium (III) oxide \\
         \hline
         (i) & \ce{NaH} & Sodium hydride \\
         \hline
         (j) & \ce{ZnCl_2} & Zinc (II) chloride \\
         \hline
         (k) & \ce{CsF} & Caesium fluoride \\
         \hline
         (l) & \ce{Li_3N} & Lithium nitride \\
         \hline
         (m) & \ce{Ag_2S} & Silver (I) sulfide \\
         \hline
         (n) & \ce{Sr_3P_2} & Strontium phosphide \\
         \hline
         (o) & \ce{MnO_2} & Manganese (IV) oxide \\
         \hline
         \hline
    \end{tabular}
\end{center}


\newpage
\subsection{Problem 8}
Write the formula for each of the following compounds:

\begin{center}
    \begin{tabular}{||c|c|c||c|c|c||}
        \hline
        \hline
        \textbf{Index} & \textbf{Name} & \textbf{Formula} & \textbf{Index} & \textbf{Name} & \textbf{Formula} \\
        \hline
        \hline
        (a) & Caesium bromide &  & (i) & Tin (II) fluoride & \\
        \hline
        (b) & Barium sulfate &  & (j) & Ammonium acetate &  \\
        \hline
        (c) & Chlorine monoxide &  & (k) & Mercury (I) chloride & \\
        \hline
        (d) & Ammonium chloride &  & (l) & Potassium cyanide &  \\
        \hline
        (e) & Silicon tetrachloride &  & (m) & Lead (IV) sulfide &  \\
        \hline
        (f) & Beryllium oxide &  & (n) & Lead (II) sulfide & \\
        \hline
        (g) & Sodium dihydrogen phosphate &  & (o) & Silicon tetrachloride & \\
        \hline
        (h) & Lithium nitride &  & (p) & Sodium peroxide &  \\
        \hline
        \hline
    \end{tabular}
\end{center}

\begin{center}
    \begin{tabular}{||c|c|c||c|c|c||}
        \hline
        \hline
        \textbf{Index} & \textbf{Name} & \textbf{Formula} & \textbf{Index} & \textbf{Name} & \textbf{Formula} \\
        \hline
        \hline
        (a) & Caesium bromide & \ce{Cs_3Br} & (i) & Tin (II) fluoride & \ce{SnF_2}\\
        \hline
        (b) & Barium sulfate & \ce{Ba(SO_4)} & (j) & Ammonium acetate & \ce{(NH_4)(C_2H_3O_2)} \\
        \hline
        (c) & Chlorine monoxide & \ce{ClO} & (k) & Mercury (I) chloride & \ce{HgCl}\\
        \hline
        (d) & Ammonium chloride & \ce{(NH_4)Cl} & (l) & Potassium cyanide & \ce{K(CN)} \\
        \hline
        (e) & Silicon tetrachloride & \ce{SiCl_4} & (m) & Lead (IV) sulfide & \ce{PbS_2} \\
        \hline
        (f) & Beryllium oxide & \ce{BaO} & (n) & Lead (II) sulfide & \ce{PbS}\\
        \hline
        (g) & Sodium dihydrogen phosphate & \ce{Na(H_2PO_4)} & (o) & Silicon tetrachloride & \ce{SiCl_4}\\
        \hline
        (h) & Lithium nitride & \ce{Li_3N} & (p) & Sodium peroxide & \ce{Na_2O_2} \\
        \hline
        \hline
    \end{tabular}
\end{center}

\newpage
\section{Problem Set 3 - Stoichiometry}
\subsection{Problem 1}
Naturally occurring sulfur consists of 4 isotopes, \ce{{}^32S} ($95 \%$), \ce{{}^33S} ($0.76 \%$), \ce{{}^34S} ($4.22 \%$), and \ce{{}^35S} (0.014 \%).
Using this data, calculate the atomic weight of naturally occurring sulfur.
The masses of the isotopes are given in the table below.

\begin{center}
    \begin{tabular}{cc}
         \textbf{Isotope} & \textbf{Atomic Mass (amu)} \\
         \ce{{}^32S} & $31.91$\\
         \ce{{}^33S} & $32.97$ \\
         \ce{{}^34S} & $33.97$ \\
         \ce{{}^35S} & $35.97$ \\
    \end{tabular}
\end{center}
We do a weighted sum to find the atomic weight.

\begin{align*}
    M &= \left (0.95 \right) \left ( 31.91 \right) + \left (0.0076 \right) \left (32.97 \right) + \left ( 0.0422 \right) \left (33.97 \right) + \left (0.00014 \right) \left (35.97 \right) \\
    &= 30. + 0.25 + 1.43 + 0.0050 \\
    &= \boxed{32} \\
\end{align*}

\newpage
\subsection{Problem 2}
A noble gas consists of 3 isotopes of masses $19.99$ amu, $20.99$ amu, and $21.99$ amu.
The relative abundance of these isotopes is $90.92\%$, $0.257\%$, and $8.82\%$, respectively.
What is the average atomic mass of this noble gas?
What element might this be?

\begin{align*}
    M &= \left ( 0.9092 \right ) \left ( 19.99\right) + \left (0.00257 \right) \left (20.99 \right) + \left ( 0.0882 \right) \left (21.99 \right) \\
    &= 18.17 + 0.0539 + 1.94 \\
    &= \boxed{20.16} \\
\end{align*}

Looking at the periodic table, it's easy to see that this element is likely \boxed{\text{Neon}}.

\newpage
\subsection{Problem 3}
An element X has 5 major isotopes, listed below along with their relative abundances.
What is this element?
Does the mass you calculated based on these data agree with that listed on your periodic table?

\begin{center}
    \begin{tabular}{||c|c|c||}
         \hline
         \hline
         \textbf{Isotope} & \textbf{\% Natural Abundance} & \textbf{Atomic Mass} \\
         \hline
         \hline
         \ce{{}^46X} & $8.0\%$ & $45.95269$ \\
         \hline
         \ce{^47X} & $7.3\%$ & $46.951764$ \\
         \hline
         \ce{^48X} & $73.8\%$ & $47.947947$ \\
         \hline
         \ce{^49X} & $5.5\%$ & $48.947841$ \\
         \hline
         \ce{^50X} & $5.4\%$ & $49.944792$ \\
         \hline
         \hline
    \end{tabular}
\end{center}



\newpage
\subsection{Problem 4}
How many moles are in $300$ atoms of nitrogen? How many grams?

\noindent We convert to moles using Avagadro's number, after noting that nitrogen is diatomic.
\begin{align*}
    n &= \left ( \frac{150\ \text{molecules}\ \ce{N_2}}{1} \right) \times \left ( \frac{1\ \text{mol}\ \ce{N_2} }{6.022 \times 10^{23}\ \text{molecules}\ \ce{N_2} } \right) \\
    &= \boxed{2.49 \times 10^{-22}\ \text{mol}\ \ce{N_2}} \\
\end{align*}
We next convert to grams using the molar mass of an \ce{N_2} molecule.

\begin{align*}
    m &= \left ( \frac{2.49 \times 10^{-22}\ \text{mol}\ \ce{N_2}}{1} \right) \times 2 \times \left ( \frac{14.0067\ \text{g}\ \ce{N_2}}{\text{mol}\ \ce{N_2}} \right) \\
    &= \boxed{6.98 \times 10^{-21}\ \text{g}\ \ce{N_2}}
\end{align*}

\newpage
\subsection{Problem 5}
If you buy $38.9$ moles of M\&M's, how many M\&M's do you have?

A mole of something is equivalent to $6.022 \times 10^{23} $ of the thing, so we simply multiply by this number to get \boxed{2.34 \times 10^{25}\ \text{M\&M's}}.

\newpage
\subsection{Problem 6}
A sample of sulfur has a mass of $8.37$ grams.
How many moles are in the sample?
How many atoms?

We divide by the molar mass of sulfur, which is 32.065 grams per mole, to get \boxed{$0.261$ moles}.
Each mole contains $6.022 \times 10^{23}$ sulfur atoms, so the sample contains $\boxed{1.57 \times 10^{23}} $ sulfur atoms.

\newpage
\subsection{Problem 7}
Give the number of moles of each elements present in 1.0 mole of each of the following substances?

\subsubsection{(a)}
\boxed{2} moles of \ce{Hg} and \boxed{2} moles of \ce{I}

\subsubsection{(b)}
\boxed{1} mole each of \ce{Li} and \ce{H}

\subsubsection{(c)}
\boxed{1} mole each of \ce{Pb} and \ce{C}, along with \boxed{3} moles of $O$

\subsubsection{(d)}
\boxed{2} moles of \ce{Ba} and \ce{As}, along with \boxed{8} moles of \ce{O}

\subsubsection{(e)}
\boxed{1} mole of \ce{Rb}, \boxed{3} moles of \ce{O}, and \boxed{5} moles of \ce{H}

\subsubsection{(f)}
\boxed{2} moles of \ce{H}, \boxed{1} mole of \ce{Si}, and \boxed{6} moles of \ce{F}

\newpage
\subsection{Problem 8}
How many grams of zinc are in $1.16 \times 10^{22}$ atoms of zinc?

\begin{align*}
    m &= \left ( \frac{1.16\times 10^{22}\ \text{atoms}}{1} \right) \times \left ( \frac{1\ \text{mol}}{6.022 \times 10^{23}\  \text{atoms}} \right) \times \left ( \frac{65.38\ \text{g}}{1\ \text{mol}} \right) \\
    &= \boxed{1.26\ \text{g}}
\end{align*}

\newpage
\subsection{Problem 9}
Calculate the molar masses of each of the following:
\subsubsection{(a)}
\ce{Cu_2SO_4}

\begin{align*}
    M &= 2 \left (63.546 \right) + 32.06 + 4 \left (15.999 \right) \\
    &= 127.092 + 32.06 + 63.996 \\
    &= \boxed{223.15} \\
\end{align*}

\subsubsection{(b)}
\ce{NH_4 OH}

\begin{align*}
    M &= 14.007 + 15.999 + 5 \left (1.008 \right) \\
    &= 14.007 + 15.999 + 5.04 \\
    &= \boxed{35.046} \\
\end{align*}

\subsubsection{(c)}
\ce{C_10H_16O}

\begin{align*}
    M &= 10 \left (12.011 \right) + 16 \left (1.008 \right) + 15.999 \\
    &= 120.11 + 16.128 + 15.999 \\
    &= \boxed{152.24}
\end{align*}

\subsubsection{(d)}
\ce{Zr(SeO_3)_2}

\begin{align*}
    M &= 91.224 + 2 \left (78.971 \right) + 6 \left (15.999 \right) \\
    &= 91.224 + 157.942 + 95.994 \\
    &= \boxed{345.160} \\
\end{align*}


\subsubsection{(e)}
\ce{Ca_2Fe(CN)_6 * 12 H_2 O}

\begin{align*}
    M &= 2 \left (40.078 \right) + 55.845 + 6 \left (12.011 \right) + 6 \left (14.007 \right) + 24 \left (1.008 \right) + 24 \left (15.999 \right) \\
    &= 80.156 + 55.845 + 72.066 + 84.042 + 24.192 + 383.976 \\
    &= \boxed{700.277} \\
\end{align*}

\subsubsection{(f)}

\ce{Cr_4(P_2O_7)_3}

\begin{align*}
    M &= 4 \left (51.996 \right) + 6 \left (30.974 \right) + 21 \left (15.999 \right) \\
    &= 207.984 + 185.844 + 335.979 \\
    &= \boxed{729.807} \\
\end{align*}

\newpage
\subsection{Problem 10}
What is the mass of $4.28 \times 10^{22}$ molecules of water?

Water is \ce{H_2 O}, so we start by finding its molar mass.
\begin{align*}
    M &= 2 \left (1.008 \right) + 15.999 \\
    &= 2.016 + 15.999 \\
    &= 18.015 \\
\end{align*}
We can now do this computation:

\begin{align*}
     m &= \left ( 4.28 \times 10^{22}\ \text{molecules} \right) \times \left ( \frac{1\ \text{mol}}{6.022 \times 10^{23}\ \text{molecules}} \right) \times \left ( \frac{18.015\ \text{g}}{\text{mol}} \right) \\
     &= \boxed{1.28\ \text{g}} \\
\end{align*}

\newpage
\subsection{Problem 11}
How many milligrams of \ce{Br_2} are in $4.8 \times 10^{20}$ molecules of \ce{Br_2}?

\begin{align*}
    m &= \left ( \frac{4.8 \times 10^{20}\ \text{molecules}}{1} \right) \left (\frac{1\ \text{mol}}{6.022 \times 10^{23}\ \text{molecules}} \right) \left ( \frac{2 \times 79.904\ \text{g}}{\text{mol}} \right) \left (\frac{1000\ \text{mg}}{\text{g}} \right)\\
    &= \boxed{130\ \text{mg}} \\
\end{align*}



\newpage
\subsection{Problem 12}
How many sodium ions are present in each of the following:

\subsubsection{(a)}
$2$ moles of sodium phosphate $\left (\ce{Na_3(PO_4)} \right)$

\begin{align*}
    N &= \left ( \frac{2\ \text{mol \ce{Na_3(PO_4)} }}{1} \right) \left ( \frac{6.022 \times 10^{23}\ \text{molecules \ce{Na_3(PO_4)} }}{1 \text{mol \ce{Na_3(PO_4)}}} \right) \left ( \frac{3\ \text{\ce{Na^+} ions}}{1 \text{moleculue \ce{Na_3(PO_4)}}} \right) \\
    &= \boxed{4 \times 10^{24}\ \text{ions \ce{Na^+}}}
\end{align*}

\subsubsection{(b)}
$5.8$ grams of sodium chloride $\left ( \ce{NaCl} \right)$

\noindent The molar mass is $22.990 + 35.45 = 58.44$.
Since the coefficient of \ce{Na} is 1, the number of sodium ions is equal to the number of molecules.

\begin{align*}
    N &= \left ( \frac{5.8\ \text{g}}{1} \right) \left ( \frac{1\ \text{mol}}{58.44\ \text{g}} \right) \left ( \frac{6.022 \times 10^{23}\ \text{ions}}{1\ \text{mol}} \right) \\
    &= \boxed{6.0\ \times 10^{22}\ \text{ions}}
\end{align*}

\subsubsection{(c)}
A mixture containing $14.2$ grams of sodium sulfate and $2.9$ grams of sodium chloride

\noindent We can find each one separately and then add them up.
Sodium sulfate is \ce{Na_2 (SO_4)} and sodium chloride is \ce{NaCl}.
The former's molar mass is

\begin{align*}
    2 \times 22.990 + 32.06 + 4 \times 15.999 &= 45.98 + 32.06 + 63.996 \\
    &= 142.04 \\
\end{align*}
For the latter, please see part (b).

\begin{align*}
    N_1 &= \frac{14.2\ \text{g \ce{Na_2 (SO_4)}} }{\text{1}} \times \frac{1\ \text{mol \ce{Na_2 (SO_4)}}}{142.04\ \text{g \ce{Na_2 (SO_4)}}} \times \frac{6.022 \times 10^{23}\ \text{molecules \ce{Na_2 (SO_4)}}}{1\ \text{mol \ce{Na_2 (SO_4)}}} \times \frac{3\ \text{\ce{Na^+} ions}}{1\ \text{molecules \ce{Na_2 (SO_4)}}} \\
    &= 1.81 \times 10^{23}\ \text{\ce{Na^+} ions} \\
    N_2 &= \frac{2.9\ \text{g \ce{NaCl}} }{\text{1}} \times \frac{1\ \text{mol \ce{NaCl}}}{58.44\ \text{g \ce{NaCl}}} \times \frac{6.022 \times 10^{23}\ \text{molecules \ce{NaCl}}}{1\ \text{mol \ce{NaCl}}} \times \frac{1\ \text{\ce{Na^+} ions}}{1\ \text{molecules \ce{NaCl}}} \\
    &= 3.0 \times 10^{22}\ \text{\ce{Na^+} ions} \\
    N &= N_1 + N_2 \\
    &= 18.1 \times 10^{22} + 3.0 \times 10^{22} \\
    &= 21.1 \times 10^{22} \\
    &= \boxed{2.11 \times 10^{23}\ \text{\ce{Na^+} ions}}
\end{align*}

\newpage
\subsection{Problem 13}
Determine the molar mass of \ce{KAl(SO_4)_2*12H_2O}.

\begin{align*}
    M &= 39.098 + 26.982 + 2 \left (32.06 + 4 \left (15.999 \right) \right) + 12 \left ( 2 \left (1.008 \right) + 15.999 \right) \\
    &= 39.098 + 26.982 + 2 \cdot 32.06 + 8 \cdot 15.999 + 24 \cdot 1.008 + 12 \cdot 15.999 \\
    &= 39.098 + 26.982 + 64.12 + 127.992 + 24.192 + 191.988 \\
    &= \boxed{474.37\ \frac{\text{g}}{\text{mol}}}\\
\end{align*}

\newpage
\subsection{Problem 14}
How many moles of cadmium (II) bromide, \ce{CdBr_2} are in a $39.25$ gram sample?

\noindent The molar mass is

\begin{align*}
    M &= 112.41 + 2 \cdot 79.904 \\
    &= 112.41 + 159.808 \\
    &= 272.218 \\
\end{align*}

\newpage
\subsection{Problem 15}
Bauxite, the principle ore used in the production of of aluminum cans, has a molecular formula of \ce{Al_2O_3*2H_2O}.

\subsubsection{(a)}
Determine the molar mass of bauxite.

\begin{align*}
    M &= 2 \cdot 26.982 + 5 \cdot 15.999 + 4 \cdot 1.008 \\
    &= 53.964 + 79.995 + 4.032 \\
    &= \boxed{137.991} \\
\end{align*}

\subsubsection{(b)}

\begin{align*}
    n &= \frac{0.58\ \text{moles bauxite}}{1} \times \frac{2\ \text{moles Al}}{1\ \text{mol bauxite}} \\
    &= 1.2\ \text{mol Al} \\
    m &= \frac{1.2\ \text{mol Al}}{1} \times \frac{26.982\ \text{g Al}}{1\ \text{mol Al}} \\
    &= \boxed{32\ \text{g Al}}
\end{align*}

\subsubsection{(c)}

\begin{align*}
    N &= 1.2\ \text{mol Al} \times \frac{6.022 \times 10^{23} \text{atoms Al}}{\text{mol Al}} \\
    &= \boxed{7.2 \times 10^{23}\ \text{atoms Al}} \\
\end{align*}

\subsubsection{(d)}

\begin{align*}
    m &= 2.1 \times 10^{24}\ \text{formula units} \times \frac{1\ \text{mol bauxite}}{6.022 \times 10^{23}\ \text{formula units }} \times \frac{137.991\ \text{g bauxite}}{1\ \text{mol bauxite}} \\
    &= \boxed{480\ \text{g bauxite}}
\end{align*}

\newpage
\subsection{Problem 16}
Calculate the mass percentage of Cl in each of the following compounds:

\subsubsection{(a)}
\ce{ClF}

\begin{align*}
    100 \times \frac{35.45}{35.45 + 18.998} &= 100 \times \frac{35.45}{54.46} \\
    &= \boxed{65.09\%} \\
\end{align*}

\subsubsection{(b)}
\ce{HClO_2}

\begin{align*}
    100 \times \frac{35.45}{35.45 + 1.008 + 2 \cdot 15.999} &= 100 \times \frac{35.45}{35.45 + 1.008 + 31.998} \\
    &= 100 \times \frac{35.45}{68.46} \\
    &= \boxed{51.78\%} \\
\end{align*}

\subsubsection{(c)}
\ce{CuCl_2}

\begin{align*}
    100 \times \frac{2 \cdot 35.45}{2 \cdot 35.45 + 63.546} &= 100 \times \frac{70.90}{134.45} \\
    &= \boxed{52.73\%} \\
\end{align*}

\subsubsection{(d)}

\ce{PuOCl}

\begin{align*}
    100 \times \frac{35.45}{244 + 35.45 + 15.999} &= 100 \times \frac{35.45}{295} \\
    &= \boxed{12.0\%} \\
\end{align*}

\newpage
\subsection{Problem 17}
Calculate the mass percentage of each element in potassium ferricyanide, \ce{K_3Fe(CN)_6}.
The molar mass is

\begin{align*}
    M &= 3 \cdot 39.098 + 55.845 + 6 \cdot 12.011 + 6 \cdot 14.007 \\
    &= 117.29 + 55.845 + 72.066 + 84.042 \\
    &= 329.24 \\
\end{align*}
The mass percentage of each element, in order of appearance, is
\begin{align*}
    100 \times \frac{3 \times 39.098}{329.24} &= \boxed{35.626\%} \\
    100 \times \frac{55.845}{329.24} &= \boxed{16.962\%} \\
    100 \times \frac{6 \times 12.011}{329.24} &= \boxed{21.889\%} \\
    100 \times \frac{6 \times 14.007}{329.24} &= \boxed{25.526\%} \\
\end{align*}

\newpage
\subsection{Problem 18}
Calculate the mass percentage of silver in each of the following compounds:

\subsubsection{(a)}
\ce{AgCl}

\begin{align*}
    100 \times \frac{107.87}{35.45 + 107.87} &= 100 \times \frac{107.87}{143.32} \\
    &= 75.265\% \\
\end{align*}

\subsubsection{(b)}
\ce{AgCN}

\begin{align*}
    100 \times \frac{107.87}{107.87 + 12.011 + 14.007} &= 100 \times \frac{107.87}{133.89} \\
    &= 80.567\% \\
\end{align*}

\subsubsection{(c)}
\ce{AgNO_3}

\begin{align*}
    100 \times \frac{107.87}{107.87 + 14.007 + 3 \cdot 15.999} &= 100 \times \frac{107.87}{107.87 + 14.007 + 47.997} \\
    &= 100 \times \frac{107.87}{169.87} \\
    &= 63.502\% \\
\end{align*}

\newpage
\subsection{Problem 19}
\subsubsection{(a)}

\begin{align*}
    \ce{Agl} + \ce{Na_2S} &\to \ce{Ag_2S} + \ce{NaI} \\
    2\ \ce{AgI} + \ce{Na_2S} &\to \ce{Ag_2S} + 2\ \ce{NaI} \\
\end{align*}

\subsubsection{(b)}

\begin{align*}
    \ce{(NH_4)_2Cr_2O_7} &\to \ce{Cr_2O_3} + \ce{N_2} + \ce{H_2O} \\
    \ce{(NH_4)_2Cr_2O_7} &\to \ce{Cr_2O_3} + 2\ \ce{N_2} + 4\ \ce{H_2O} \\
\end{align*}

\subsubsection{(c)}

\begin{align*}
    \ce{Na_3PO_4} + \ce{HCl} &\to \ce{NaCl} + \ce{H_3PO_4} \\
    \ce{Na_3PO_4} + 3\ \ce{HCl} &\to 3\ \ce{NaCl} + \ce{H_3PO_4} \\
\end{align*}

\subsubsection{(d)}

\begin{align*}
    \ce{TiCl_4} + \ce{H_2O} &\to \ce{TiO_2} + \ce{HCl} \\
    \ce{TiCl_4} + 2\ \ce{H_2O} &\to \ce{TiO_2} + 4\ \ce{HCl} \\
\end{align*}

\subsubsection{(e)}

\begin{align*}
    \ce{Ba_3N_2} + \ce{H_2O} &\to \ce{Ba(OH)_2} + \ce{NH_3} \\
    \ce{Ba_3N_2} + 6\ \ce{H_2O} &\to 3\ \ce{Ba(OH)_2} + 2\ \ce{NH_3} \\
\end{align*}

\subsubsection{(f)}

\begin{align*}
    \ce{HNO_2} &\to \ce{HNO_3} + \ce{NO} + \ce{H_2O} \\
\end{align*}
For this one, we setup a system.
\begin{align*}
    a\ \ce{HNO_2} &\to b\ \ce{HNO_3} + c\ \ce{NO} + d\ \ce{H_2O} \\
\end{align*}

\begin{align}
    a &= b + 2d \\
    a &= b + c \\
    2a &= 3b + c + d
\end{align}

Comparing (1) and (2), we see that $c=2d$.
Substituting into (3), we see that $2a = 3b + 3c$.
Let $d = 1$, so $c = 2 \cdot 1 = 2$.
Substituting, we have

\begin{align*}
    a &= b + 2 \\
    2a &= 3b + 3 \\
\end{align*}
We can solve this system to get $(a, b) = (3, 1)$.
Finally,
\begin{align*}
    3\ \ce{HNO_2} &\to  \ce{HNO_3} + 2\ \ce{NO} + \ce{H_2O} \\
\end{align*}

\newpage
\subsection{Problem 20}
Balance the following equation:
\begin{align*}
    \ce{NH_4OH}\ (\ell) + \ce{KAl(SO_4)_2*12H_2O} &\to \ce{Al(OH)_3}\ (s) + \ce{(NH_4)_2(SO_4)} + \ce{KOH}\ (aq) + \ce{H_2O}\ (\ell) \\
    4\ \ce{NH_4OH}\ (\ell) + \ce{KAl(SO_4)_2*12H_2O} &\to \ce{Al(OH)_3}\ (s) + 2\ \ce{(NH_4)_2(SO_4)} + \ce{KOH}\ (aq) + 12\ \ce{H_2O}\ (\ell) \\
\end{align*}

\newpage
\subsection{Problem 21}
Balance the following equation:
\begin{align*}
    \ce{Fe}\ (s) + \ce{HC_2H_3O_2}\ (aq) &\to \ce{Fe(C_2H_3O_2)}\ (aq) + \ce{H_2}\ (g) \\
    2\ \ce{Fe}\ (s) + 2\ \ce{HC_2H_3O_2}\ (aq) &\to 2\ \ce{Fe(C_2H_3O_2)}\ (aq) + \ce{H_2}\ (g) \\
\end{align*}

\newpage
\subsection{Problem 22}
How many grams of sodium hydroxide are required to form $51.63$ grams of lead hydroxide?

\begin{align*}
    \ce{Pb(NO_3)_2} + \ce{NaOH} &\to \ce{Pb(OH)_2} + \ce{NaNO_3} \\
    2\ \ce{Pb(NO_3)_2} + 6\ \ce{NaOH} &\to 3\ \ce{Pb(OH)_2} + 6\ \ce{NaNO_3} \\
\end{align*}
We first find the molar mass of each compound.

\begin{align*}
    M_{\ce{Pb(OH)_2}} &= 207.2 + 2 \cdot 1.008 + 2 \cdot 15.999 \\
    &= 207.2 + 2.016 + 31.998 \\
    &= 241.2 \\
\end{align*}

\begin{align*}
    M_{\ce{NaOH}} &= 22.990 + 15.999 + 1.008 \\
    &= 39.997 \\
\end{align*}

Finally,

\begin{align*}
    m_{\ce{NaOH}} &= 51.63\ \text{g \ce{Pb(OH)_2}} \times \frac{1\ \text{mol \ce{Pb(OH)_2}}}{241.2\ \text{g \ce{Pb(OH)_2}}} \times \frac{6\ \text{mol \ce{NaOH}}}{3\ \text{mol \ce{Pb(OH)_2}}} \times \frac{39.997\ \text{g \ce{NaOH}}}{1\ \text{mol \ce{NaOH}}} \\
    &= \boxed{17.12\ \text{g \ce{NaOH}}}
\end{align*}

\newpage
\subsection{Problem 23}
How may grams of water vapor can be generated from the combustion of $18.74$ grams of ethanol?

\begin{align*}
    \ce{C_2H_6O} + \ce{O_2} &\to \ce{C_2O} + \ce{H_2O} \\
     \ce{C_2H_5O} + \ce{O_2} &\to \ce{C_2O} + 3\ \ce{H_2O} \\
\end{align*}

Next, we compute the molar mass of ethanol and water vapor:

\begin{align*}
    M_{\ce{C_2H_6O}} &= 2 \cdot 12.011 + 6 \cdot 1.008 + 15.999 \\
    &= 12.011 + 6.048 + 15.999 \\
    &= 34.058 \\
    M_{\ce{H_2O}} &= 2 \cdot 1.008 + 15.999 \\
    &= 1.016 + 15.999 \\
    &= 17.015 \\
\end{align*}

\begin{align*}
    m_{\ce{H_2O}} &= 18.74\ \text{g \ce{C_2H_6O}} \times \frac{1\ \text{mol \ce{C_2H_6O}}}{34.058\ \text{g \ce{C_2H_6O}} } \times \frac{3\ \text{mol\ \ce{H_2O}}}{1\ \text{mol \ce{C_2H_6O}}} \times \frac{17.015\ \text{g \ce{H_2O}}}{1\ \text{mol \ce{H_2O}}} \\
    &= \boxed{28.09\ \text{g \ce{H_2O}}} \\
\end{align*}

\newpage
\subsection{Problem 24}
How many grams of potassium iodide are necessary to completely react with 20.61 g of mercury (II) chloride?

\begin{align*}
    \ce{HgCl_2} + \ce{KI} &\to \ce{HgI_2} + \ce{KCl} \\
    \ce{HgCl_2} + 2\ \ce{KI} &\to \ce{HgI_2} + 2\ \ce{KCl} \\
\end{align*}

\noindent The molar mass of potassium iodide is $39.098 + 126.90 = 166.00$, while the molar mass of mercury (II) chloride is $2 \cdot 35.45 + 200.59 = 70.90 + 200.59 = 271.49$.

\begin{align*}
    m_{\ce{KI}} &= 20.61\ \text{g \ce{HgCl_2}} \times \frac{1\ \text{mol \ce{HgCl_2}}}{271.49\ \text{g \ce{HgCl_2}}} \times \frac{2\ \text{mol \ce{KI}}}{1\ \text{mol \ce{HgCl_2}}} \times \frac{166.00\ \text{g \ce{KI}}}{1\ \text{mol \ce{KI}}} \\
    &= \boxed{25.20\ \text{g \ce{KI}}} \\
\end{align*}

\newpage
\subsection{Problem 25}
How many grams of oxygen are required to completely react with 22.8 grams of methane, \ce{CH_4}.

\begin{align*}
    \ce{CH_4} + \ce{O_2} &\to \ce{CO_2} + \ce{H_2O} \\
    \ce{CH_4} + 2\ \ce{O_2} &\to \ce{CO_2} + 2\ \ce{H_2O} \\
\end{align*}
The molar masses of methane and diatomic oxygen gas are, respectively,
\begin{align*}
    M_{\ce{CH_4}} &= 12.011 + 4 \cdot 1.008 \\
    &= 12.011 + 4.032 \\
    &= 16.043 \\
    M_{\ce{O_2}} &= 2 \cdot 15.999 \\
    &= 31.998 \\
\end{align*}
We can now compute:
\begin{align*}
    m_{\ce{O_2}}&= 22.8\ \text{g \ce{CH_4}} \times \frac{1\ \text{mol \ce{CH_4}}}{16.043\ \text{g \ce{CH_4} }} \times \frac{2\ \text{mol \ce{O_2}}}{1\ \text{mol \ce{CH_4}}} \times \frac{31.998\ \text{g \ce{O_2}}}{1 \text{mol \ce{O_2}}} \\
    &= \boxed{90\ \text{g \ce{O_2}}} \\
\end{align*}


\subsection{Problem 26}
If, in the previous problem, only $25.9$ grams of water vapor were formed, how many grams of methane actually reacted with oxygen?

\noindent We first find the molar mass of water vapor:

\begin{align*}
    M_{\ce{H_2O}} &= 2 \cdot 1.008 + 15.999 \\
    &= 1.016 + 15.999 \\
    &= 17.015 \\
\end{align*}

\noindent Now, we can compute:
\begin{align*}
    m_{\ce{CH_4}} &= 25.9\ \text{g \ce{H_2O}} \times \frac{1\ \text{mol \ce{H_2O}}}{17.015\ \text{g \ce{H_2O}}} \times \frac{1\ \text{mol \ce{CH_4}}}{2\ \text{mol \ce{H_2O}}} \times \frac{16.043\ \text{g \ce{CH_4}}}{1\ \text{mol \ce{CH_4}}}\\
    &= \boxed{12.2\ \text{g \ce{CH_4}}}
\end{align*}

\newpage
\subsection{Problem 27}
What mass of calcium carbonate, \ce{CaCO_3}, would be formed if $248.6$ grams of carbon dioxide, \ce{CO_2}, were exhaled into limewater, \ce{Ca(OH)_2}?
How many grams of calcium would be needed to form that amount of calcium carbonate?
Assume 100\% yield in each reaction.
\subsubsection{(a)}
\begin{align*}
    \ce{Ca(OH)_2} + \ce{CO_2} &\to \ce{CaCO_3} + \ce{H_2O} \\
\end{align*}
This is balanced already, so we don't need to do anything in that regard.
However, we still need to find the atomic masses.


\begin{align*}
    M_{\ce{CO_2}} &= 12.011 + 2 \cdot 15.999 \\
    &= 12.011 + 31.998 \\
    &= 44.009 \\
    M_{\ce{CaCO_3}} &= 12.011 + 40.078 + 3 \cdot 15.999 \\
    &= 12.011 + 40.078 + 47.997 \\
    &= 100.086 \\
    m_{\ce{CO_2}} &= 248.6\ \text{g \ce{CO_2}} \times \frac{1\ \text{mol \ce{CO_2}}}{44.009\ \text{g \ce{CO_2}}} \times \frac{1\ \text{mol \ce{CaCO_3}}}{1\ \text{mol \ce{CO_2}}} \times \frac{100.086\ \text{g \ce{CaCO_3}}}{1\ \text{mol \ce{CaCO_3}}} \\
    &= \boxed{565.4\ \text{g \ce{CaCO_3}}} \\
\end{align*}

\newpage
\subsection{Problem 28}
The following reaction is used to form lead iodide crystals. What mass of crystal $\left (\ce{PbI_2} \right)$ could be formed from $1.0 \times 10^{3}$ grams of lead (II) acetate $\left [ \ce{Pb(C_2H_3O_2)_2} \right]$?
\begin{align*}
    \ce{Pb(C_2H_3O_2)_2} + 2\ \ce{KI} &\to \ce{PbI_2} + 2\ \ce{KC_2H_3O_2} \\
\end{align*}
We find the atomic masses of each compound and compute:
\begin{align*}
    M_{\ce{Pb(C_2H_3O_2)_2}} &= 207.2 + 4 \cdot 12.011 +  6 \cdot 1.008 + 4 \cdot 15.999 \\
    &= 207.2 + 48.044 + 6.048 + 63.996 \\
    &= 319.2 \\
    M_{\ce{PbI_2}} &= 207.2 + 2\cdot 126.9045 \\
    &= 207.2 + 253.8090 \\
    &= 461.0 \\
    m_{\ce{PbI_2}} &= 1.0 \times 10^{3}\ \text{g \ce{Pb(C_2H_3O_2)_2}} \times \frac{1\ \text{mol \ce{Pb(C_2H_3O_2)_2}}}{319.2\ \text{g \ce{Pb(C_2H_3O_2)_2}}} \times \frac{1\ \text{mol \ce{PbI_2}}}{1\ \text{mol \ce{Pb(C_2H_3O_2)_2}} } \times \frac{461.0\ \text{g \ce{PbI_2}}}{1\ \text{mol \ce{PbI_2}} } \\
    &= \boxed{1400\ \text{g \ce{PbI_2}}} \\
\end{align*}

\newpage
\subsection{Problem 29}
You were hired by a laboratory to recycle $6$ moles of silver ions.
You were given $150.$ grams of copper.
How many grams of silver can you recover using the following reaction?
Is this enough copper to recycle $6$ moles of silver ions?

\begin{align*}
    2\ \ce{Ag^+} + \ce{Cu} &\to 2\ \ce{Ag} + \ce{Cu^+} \\
\end{align*}

\noindent We compute:
\begin{align*}
    n_{\ce{Ag}} &= 150.\ \text{g \ce{Cu}} \times \frac{1\ \text{mol \ce{Cu}}}{63.546\ \text{g \ce{Cu}}} \times \frac{2\ \text{mol \ce{Ag}}}{1\ \text{mol \ce{Cu}}} \\
    &= \boxed{4.72\ \text{mol \ce{Ag}}} \\
\end{align*}

We can convert this into grams by multiplying by the molar mass.
\begin{align*}
    m_{\ce{Ag}} &= 4.72\ \text{mol \ce{Ag}} \times \frac{107.8682\ \text{g \ce{Ag}}}{1\ \text{mol \ce{Ag}}} \\
    &= \boxed{509\ \text{g \ce{Ag}}} \\
\end{align*}



\newpage
\subsection{Problem 30}
If 42.7 grams \ce{Cr_2O_3} and 9.8 grams of \ce{Al} are mixed and reacted until one of the reactants is used via a single replacement reaction.

\subsubsection{(a)}
Which reactant is the limiting reagent and which is in excess?

\noindent First we find the equation and balance it:
\begin{align*}
    \ce{Cr_2O_3} + \ce{Al} &\to \ce{Al_2O_3} + \ce{Cr} \\
    \ce{Cr_2O_3} + 2\ \ce{Al} &\to \ce{Al_2O_3} + 2\ \ce{Cr} \\
\end{align*}

\noindent Next, we must find the molar mass of \ce{Cr_2O_3}:
\begin{align*}
    M_{\ce{Cr_2O_3}} &= 2 \cdot 51.9961 + 3 \cdot 15.999 \\
    &= 103.992 + 47.997 \\
    &= 151.989 \\
\end{align*}

\noindent Next, we must find the number of moles of each that will form:
\begin{align*}
    n_{\ce{Cr_2O_3}} &= \frac{42.7\ \text{g \ce{Cr_3O_3}}}{\frac{151.989\ \text{g \ce{Cr_2O_3}}}{1\ \text{mol \ce{Cr_2O_3}}}} \\
    &= 0.276\ \text{mol \ce{Cr_2O_3}} \\
    n_{\ce{Al}} &= \frac{9.8\ \text{g \ce{Al}}}{\frac{26.98154\ \text{g \ce{Al}}}{1\ \text{mol \ce{Al}}}} \\
    &= 0.36\ \text{mol \ce{Al}} \\
\end{align*}
Because of the 1:2 molar ratio between the chromium (III) oxide and the aluminum, we can see that the aluminum is the limiting reagent and the chromium (III) oxide is in excess.

\subsubsection{(b)}
How many grams of chromium will be formed?

\begin{align*}
    m_{\ce{Cr}} &= 0.36\ \text{mol \ce{Al}} \times \frac{1 \ \text{mol \ce{Cr}} }{1\ \text{mol \ce{Al}}} \times \frac{51.9961\ \text{g \ce{Cr}}}{1\ \text{mol \ce{Cr}}} \\
    &= \boxed{19\ \text{g \ce{Cr}}} \\
\end{align*}


\subsubsection{(c)}
How much excess reactant will be left?

\noindent We can calculate the moles of chromium (III) oxide used by simply halving the number of moles of aluminum used, so this will be $n_{\ce{Cr_2O_3}} = 0.5 \cdot 0.36 = 0.18 \text{mol \ce{Al}}$.
This means that $0.276-0.18 = \boxed{0.10}$ moles of the chromium are unused.

















\end{document}
